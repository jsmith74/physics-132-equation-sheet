\documentclass[10pt,landscape,a4paper]{article}
\usepackage[utf8]{inputenc}

\usepackage{tikz}
\usetikzlibrary{shapes,positioning,arrows,fit,calc,graphs,graphs.standard}
%\usepackage[nosf]{kpfonts}
\usepackage[t1]{sourcesanspro}
%\usepackage[lf]{MyriadPro}
%\usepackage[lf,minionint]{MinionPro}
\usepackage{multicol}
\usepackage{wrapfig}
\usepackage[top=4mm,bottom=14mm,left=4mm,right=4mm]{geometry}
\usepackage[framemethod=tikz]{mdframed}
\usepackage{microtype}

\let\bar\overline

\definecolor{myblue}{cmyk}{0,0,0,100}

\def\firstcircle{(0,0) circle (1.5cm)}
\def\secondcircle{(0:2cm) circle (1.5cm)}

\colorlet{circle edge}{myblue}
\colorlet{circle area}{myblue!5}

\tikzset{filled/.style={fill=circle area, draw=circle edge, thick},
	outline/.style={draw=circle edge, thick}}

\pgfdeclarelayer{background}
\pgfsetlayers{background,main}

\everymath\expandafter{\the\everymath \color{myblue}}
\everydisplay\expandafter{\the\everydisplay \color{myblue}}

\renewcommand{\baselinestretch}{.8}
\pagestyle{empty}

\global\mdfdefinestyle{header}{%
	linecolor=gray,linewidth=1pt,%
	leftmargin=0mm,rightmargin=0mm,skipbelow=0mm,skipabove=0mm,
}

\newcommand{\header}{
	\begin{mdframed}[style=header]
		\footnotesize
		\sffamily
		Equation Sheet for Physics 132
	\end{mdframed}
}

\makeatletter
\renewcommand{\section}{\@startsection{section}{1}{0mm}%
	{.2ex}%
	{.2ex}%x
	{\color{myblue}\sffamily\small\bfseries}}
\renewcommand{\subsection}{\@startsection{subsection}{1}{0mm}%
	{.2ex}%
	{.2ex}%x
	{\sffamily\bfseries}}



\def\multi@column@out{%
	\ifnum\outputpenalty <-\@M
	\speci@ls \else
	\ifvoid\colbreak@box\else
	\mult@info\@ne{Re-adding forced
		break(s) for splitting}%
	\setbox\@cclv\vbox{%
		\unvbox\colbreak@box
		\penalty-\@Mv\unvbox\@cclv}%
	\fi
	\splittopskip\topskip
	\splitmaxdepth\maxdepth
	\dimen@\@colroom
	\divide\skip\footins\col@number
	\ifvoid\footins \else
	\leave@mult@footins
	\fi
	\let\ifshr@kingsaved\ifshr@king
	\ifvbox \@kludgeins
	\advance \dimen@ -\ht\@kludgeins
	\ifdim \wd\@kludgeins>\z@
	\shr@nkingtrue
	\fi
	\fi
	\process@cols\mult@gfirstbox{%
		%%%%% START CHANGE
		\ifnum\count@=\numexpr\mult@rightbox+2\relax
		\setbox\count@\vsplit\@cclv to \dimexpr \dimen@-1cm\relax
		\setbox\count@\vbox to \dimen@{\vbox to 1cm{\header}\unvbox\count@\vss}%
		\else
		\setbox\count@\vsplit\@cclv to \dimen@
		\fi
		%%%%% END CHANGE
		\set@keptmarks
		\setbox\count@
		\vbox to\dimen@
		{\unvbox\count@
			\remove@discardable@items
			\ifshr@nking\vfill\fi}%
	}%
	\setbox\mult@rightbox
	\vsplit\@cclv to\dimen@
	\set@keptmarks
	\setbox\mult@rightbox\vbox to\dimen@
	{\unvbox\mult@rightbox
		\remove@discardable@items
		\ifshr@nking\vfill\fi}%
	\let\ifshr@king\ifshr@kingsaved
	\ifvoid\@cclv \else
	\unvbox\@cclv
	\ifnum\outputpenalty=\@M
	\else
	\penalty\outputpenalty
	\fi
	\ifvoid\footins\else
	\PackageWarning{multicol}%
	{I moved some lines to
		the next page.\MessageBreak
		Footnotes on page
		\thepage\space might be wrong}%
	\fi
	\ifnum \c@tracingmulticols>\thr@@
	\hrule\allowbreak \fi
	\fi
	\ifx\@empty\kept@firstmark
	\let\firstmark\kept@topmark
	\let\botmark\kept@topmark
	\else
	\let\firstmark\kept@firstmark
	\let\botmark\kept@botmark
	\fi
	\let\topmark\kept@topmark
	\mult@info\tw@
	{Use kept top mark:\MessageBreak
		\meaning\kept@topmark
		\MessageBreak
		Use kept first mark:\MessageBreak
		\meaning\kept@firstmark
		\MessageBreak
		Use kept bot mark:\MessageBreak
		\meaning\kept@botmark
		\MessageBreak
		Produce first mark:\MessageBreak
		\meaning\firstmark
		\MessageBreak
		Produce bot mark:\MessageBreak
		\meaning\botmark
		\@gobbletwo}%
	\setbox\@cclv\vbox{\unvbox\partial@page
		\page@sofar}%
	\@makecol\@outputpage
	\global\let\kept@topmark\botmark
	\global\let\kept@firstmark\@empty
	\global\let\kept@botmark\@empty
	\mult@info\tw@
	{(Re)Init top mark:\MessageBreak
		\meaning\kept@topmark
		\@gobbletwo}%
	\global\@colroom\@colht
	\global \@mparbottom \z@
	\process@deferreds
	\@whilesw\if@fcolmade\fi{\@outputpage
		\global\@colroom\@colht
		\process@deferreds}%
	\mult@info\@ne
	{Colroom:\MessageBreak
		\the\@colht\space
		after float space removed
		= \the\@colroom \@gobble}%
	\set@mult@vsize \global
	\fi}

\makeatother
\setlength{\parindent}{0pt}

\begin{document}
	\begin{multicols*}{5}
		\section*{\large Physical Constants}
		\subsection*{Gravity Acceleration on Earth:}
		$$ g = 9.81 \enspace \frac{\textrm{m}}{\textrm{s}^2} $$
		\subsection*{Mass of Proton, Neutron}
		$$m_p = m_n = 1.67 \times 10^{-27} \enspace \textrm{kg}$$
		\subsection*{Mass of Electron}
		$$m_e = 9.11 \times 10^{-31} \enspace \textrm{kg}$$
		\subsection*{Coulomb's Law Constant}
		$$ k = \frac{1}{4 \pi \epsilon_0} = 8.99 \times 10^9 \enspace \frac{\textrm{N} \textrm{m}^2}{\textrm{C}^2}$$
		\subsection*{Permittivity of Free Space}
		$$ \epsilon_0 = 8.85 \times 10^{-12} \enspace \frac{\textrm{C}^2}{\textrm{N} \textrm{m}^2}$$
		\subsection*{Permeability of Free Space}
		$$ \mu_0 = 1.26 \times 10^{-6} \enspace \frac{\textrm{T m}}{\textrm{A}} $$
		\subsection*{Fundamental Unit of Charge}
		$$ e = 1.60 \times 10^{-19} \enspace \textrm{C}$$
		\subsection*{Speed of Light in Vacuum}
		$$ c = 3.0 \times 10^{8} \enspace \frac{\textrm{m}}{\textrm{s}} $$
		\subsection*{Planck's Constant}
		$$ \small h = 6.63 \times 10^{-34} \enspace \textrm{Js} $$
		$$ = 4.14 \times 10^{-15} \enspace \textrm{eVs} $$
		\subsection*{Bohr radius}
		$$ a_B = 5.29 \times 10^{-11} \enspace \textrm{m}$$
		\subsection*{Quadratic Equation}
		$$x = \frac{-b \pm \sqrt{b^2 - 4 ac}}{2 a}$$
		\subsection*{ Small Angle Approximation}
		$$\mbox{if} \quad \theta << 1 \quad \mbox{radian:}$$
		$$ \sin\theta \approx \tan\theta \approx \theta  $$
		$$\cos\theta \approx 1$$
		\subsection*{Volume of a Sphere}
		$$V = \frac{4}{3} \pi r^3$$
		\subsection*{Surface Area of a Sphere}
		$$ A = 4 \pi r^2 $$
		\section*{\large Electrostatics}
		$$ \vec{F} = \vec{E} q \quad \quad \vec{F} = m \vec{a}$$
		\subsection*{Coulomb's Law}
		$$ F = k \frac{q_1 q_2}{r^2} \hat{r}$$
		\subsection*{E Field Due to:}
		$$ \vec{E} = \frac{1}{4 \pi \epsilon_0} \frac{q}{r^2} \hat{r} \quad  \mbox{(point charge)}$$
		$$ \vec{E} = \frac{1}{4 \pi \epsilon_0} \frac{2 \vec{p}}{r^3} \quad \mbox{(dipole, on axis)} $$
		$$ \small \vec{E} = - \frac{1}{4 \pi \epsilon_0} \frac{\vec{p}}{r^3} \quad \mbox{(dipole, bisecting plane)}$$
		$$ \vec{p} = q \vec{s}$$
		$$ |\vec{E}| = \frac{1}{4 \pi \epsilon_0} \frac{2 |\lambda|}{r} \quad \mbox{(line of charge)}$$
		$$ |\vec{E}| = \frac{\eta}{2 \epsilon_0} \quad \mbox{(plane of charge)}$$
		$$ \vec{E} = \frac{Q}{4 \pi \epsilon_0 r^2} \hat{r} \quad \mbox{(sphere of charge)}$$
		$$\vec{E} = \frac{\eta}{\epsilon_0} = \frac{Q}{\epsilon_0 A} \quad \mbox{(capacitor)}$$
		$$ \small E_z = \frac{\eta}{2 \epsilon_0} (1-\frac{z}{\sqrt{z^2+R^2}}) \quad \mbox{(disc)}$$
		\subsection*{Electric Flux}
		$$ \Phi_e = \int \vec{E} \cdot d\vec{A}$$
		\subsection*{Gauss' Law}
		$$ \Phi_e = \frac{Q_{in}}{\epsilon_0} $$
		\subsection*{Electric Potential}
		$$ E_s = -\frac{dV}{ds} $$
		$$ \Delta V = V_f - V_i = - \int_{s_i}^{s_f} E_s ds $$ $$ = - \int_{i}^{f} \vec{E} \cdot d\vec{s}$$
		$$ V = E s \quad \mbox{(inside capacitor)}$$
		$$ V = \frac{1}{4 \pi \epsilon_0} \frac{q}{r} \quad \mbox{(around point charge)}$$
		$$ \Delta V = I R \quad \mbox{(across resistor)} $$
		\subsection*{Electric Potential Energy}
		$$ U = q V$$
		$$ \Delta K + \Delta U = 0$$
		$$\small  U = U_0 + q E s \quad \mbox{(charge in uniform field)} $$
		$$ U = \frac{1}{4 \pi \epsilon_0} \frac{q_1 q_2}{r} \quad \mbox{(two point charges)} $$
		$$ U = -\vec{p} \cdot \vec{E} \quad \mbox{(dipole)}$$
		\section*{\large Circuits}
		\subsection*{Kirchhoff's Junction Law}
		$$ \sum I_{in} = \sum I_{out} $$
		\subsection*{Kirchhoff's Loop Law}
		$$ \sum_{\mbox{\small n in loop}} (\Delta V_n) = 0$$
		\subsection*{Voltage Across Components: }
		$$ \Delta V = \pm \varepsilon  \quad \mbox{(battery)} $$
		$$ \Delta V = \pm I R \quad \mbox{(resistor)}$$
		$$ \Delta V = \pm \frac{Q}{C} \quad \mbox{(capacitor)}$$
		$$ \Delta V = \pm L \frac{dI}{dt} \quad \mbox{(inductor)}$$
		\subsection*{Resistors in Series}
		$$\Delta V_{eqv} = \Delta V_1 + \Delta V_2 + \dots + \Delta V_N$$
		$$ I_{in} = I_1 = I_2 = \dots = I_N$$
		$$ R_{eqv} = R_1 + R_2 + \dots + R_N $$
		\subsection*{Resistors in Parallel}
		$$ \Delta V_{eqv} = \Delta V_1 = \Delta V_2 = \dots = \Delta V_N $$
		$$ I_{in} = I_1 + I_2 + \dots + I_N $$
		$$\frac{1}{R_{eqv}} = \frac{1}{R_1} + \frac{1}{R_2} + \dots + \frac{1}{R_N}$$
		\subsection*{Capacitors in Series}
		$$\Delta V_{eqv} = \Delta V_1 + \Delta V_2 + \dots + \Delta V_N$$
		$$ Q_{eqv} = Q_1 = Q_2 = \dots = Q_N$$
		$$\frac{1}{C_{eqv}} = \frac{1}{C_1} + \frac{1}{C_2} + \dots + \frac{1}{C_N}$$
		\subsection*{Capacitors in Parallel}
		$$ \Delta V_{eqv} = \Delta V_1 = \Delta V_2 = \dots = \Delta V_N $$
		$$ Q_{eqv} = Q_1 + Q_2 + \dots + Q_N $$
		$$ C_{eqv} = C_1 + C_2 + \dots + C_N $$
		\subsection*{Power and Energy}
		$$ P = I \varepsilon \quad \mbox{(battery)}$$
		$$ P = I \Delta V_R  = \frac{(\Delta V_R)^2}{R} \quad \mbox{(Resistor)}$$
		$$ U = \frac{1}{2} C (\Delta V)^2 \quad \mbox{(Capacitor)}$$
		$$ U = \frac{1}{2} L I^2 \quad \mbox{(inductor)} $$
		$$ \small u_B = \frac{1}{2 \mu_0} B^2 \enspace \mbox{(inductor energy density)}$$
				\subsection*{Inductance of Solenoid}
				$$ L = \frac{\mu_0 N^2 A}{l} $$
				\subsection*{LC Circuits}
				$$ \omega = \sqrt{\frac{1}{L C}}$$
				\subsection*{LR Circuits}
				$$ \tau = \frac{L}{R} $$
		\section*{\large Current and Resistance}
		$$ N_e = \eta_e V = \eta_e A \Delta x = \eta_e A v_d \Delta t$$
		$$ i_e = \eta_e A v_d = \frac{\eta_e e \tau A}{m} E$$
		$$v_d = \frac{e \tau}{m} E$$
		$$ I = e i_e = \eta_e e v_d A $$
		$$ J = \frac{I}{A} = \eta_e e v_d$$
		$$ J = \sigma E $$
		$$ \Delta V_{wire} = I R $$
		$$ \sigma = \frac{\eta_e e^2 \tau}{m} $$
		$$ \rho = \frac{1}{\sigma} \quad \quad R = \frac{\rho L}{A}$$
		$$ E_{wire} = \frac{\Delta V_{wire}}{L} $$

		\section*{\large Magnetic Fields}
		$$ \vec{F} = q(\vec{v} \times \vec{B}) = q v B \sin(\alpha) \hat{n} $$
		$$ \vec{F} = I (\vec{l} \times \vec{B}) $$
		\subsection*{B Field Due To:}
		$$ \vec{B} = \frac{\mu_0}{4 \pi} \frac{q \vec{v} \times \hat{r}}{r^2} \quad \mbox{(point charge)}$$
		$$ \small \vec{B} = \frac{\mu_0}{4 \pi} \frac{I \Delta \vec{s} \times \hat{r}}{r^2} \quad \mbox{(segment of current)}$$
		$$ |\vec{B}| = \frac{\mu_0 N I}{2 R} \quad \mbox{(coil, in center)} $$
		$$ |\vec{B}| = \mu_0 n I \quad \mbox{(solenoid, inside)}$$
		$$ |\vec{B}| = \frac{\mu_0 I}{2 \pi d} \quad \mbox{(wire, infinite)}$$
		\subsection*{Magnetic Flux}
		$$\Phi_m = \int \vec{B} \cdot d \vec{A}$$
		$$ \small  \Phi_m = \vec{B} \cdot \vec{A} = |\vec{B}||\vec{A}|\cos\theta \quad \mbox{(uniform)} $$
		\subsection*{Faraday's Law}
		$$ \varepsilon = N \left| \frac{d \Phi_m}{dt} \right| $$
		\subsection*{Ampere's Law}
		$$ \oint \vec{B} \cdot d\vec{s} = \mu_0 I_{through}$$ 
		\section*{\large Electromagnetic Fields}
		$$v_{em} = c = \frac{1}{\sqrt{\epsilon_0 \mu_0}}$$
		$$ E = c B$$
		$$ I = \frac{P}{A} = \frac{1}{2 c \mu_0} E_0^2$$
		\subsection*{Poynting Vector}
		$$ \vec{S} = \frac{1}{\mu_0} (\vec{E} \times \vec{B})$$
		\subsection*{ Malus' Law}
		$$ I = I_0 \cos^2 \theta $$
		\section*{\large Maxwell's Equations}
		$$ \oint \vec{E} \cdot d \vec{A} = \frac{Q_{in}}{\epsilon_0} $$
		$$ \oint \vec{B} \cdot d \vec{A} = 0$$
		$$ \oint \vec{E} \cdot d \vec{s} = \frac{-d \Phi_m}{dt} $$
		$$ \oint \vec{B} \cdot d \vec{s} = \mu_0 I_{through} + \epsilon_0 \mu_0 \frac{d \Phi_e}{dt}$$
		\subsection*{Lorentz Force Law}
		$$ \vec{F} = q (\vec{E} + \vec{v} \times \vec{B}) $$
		\section*{\large AC Circuits}
		$$ \varepsilon = \varepsilon_0 \cos \omega t \quad \quad \omega = 2 \pi f $$
		$$ V_{rms} = \frac{V}{\sqrt{2}}, \quad I_{rms} = \frac{I}{\sqrt{2}} $$
		$$ \varepsilon_{rms} = \frac{\varepsilon_0}{\sqrt{2}} $$
		\subsection*{Resistors}
		$$\mbox{i and v: In Phase}$$
		$$ V_R = I R $$
		$$ P_{avg} = I_{rms} V_{rms} $$
		\subsection*{Capacitors}
		$$ \mbox{i leads v by } 90^\circ $$
		$$ V_C = I X_C \quad \quad X_C = \frac{1}{\omega C} $$
		$$ P_{avg} = 0 $$
		\subsection*{Inductors}
		$$ \mbox{i lags v by } 90^\circ $$
		$$ V_L = I X_L \quad \quad X_L = \omega L $$
		$$ P_{avg} = 0 $$
		\subsection*{RC Filter Circuits}
		$$ V_C = \frac{\varepsilon_0 X_C}{\sqrt{R^2 + X_C^2}} $$
		$$ V_R = \frac{\varepsilon_0 R}{\sqrt{R^2 + X_C^2}} $$
		\subsection*{Series RLC Circuits}
		$$ I = \frac{\varepsilon_0}{Z} $$
		$$ Z = \sqrt{R^2 + (X_L-X_C)^2} $$
		$$ V_R = I R \quad V_L = I X_L $$
		$$ V_C = I X_C $$
		$$ \omega_0 =1/\sqrt{L C}$$
		$$ \phi = \tan^{-1} ((X_L-X_C)/R)$$
		$$ P_{source} = I_{rms} \varepsilon_{rms} \cos \phi $$
		$$ P_R = I_{rms} V_{rms} = I^2_{rms} R $$
		\section*{\large Wave Optics}
		\subsection*{Young's Double Slit Experiment}
		$$ \theta_m = m \frac{\lambda}{d} \quad \quad m=0,1,2,3,\dots $$
		$$ y_m = \frac{m \lambda L}{d} \quad \quad m=0,1,2,3,\dots $$
		$$ I_{double} = 4 I_1 \cos^2(\frac{\pi d}{\lambda L} y) $$
		\subsection*{Diffraction Grating}
		$$ d \sin \theta_m = m \lambda \quad \quad m=0,1,2,3,\dots $$
		$$ y_m = L \tan \theta_m $$
		\subsection*{Single-Slit Diffraction}
		$$ \theta_p = p \frac{\lambda}{a} \quad \quad p=1,2,3,\dots$$
		$$ y_p = \frac{p \lambda L}{a} \quad \quad p=1,2,3,\dots$$
		$$ w= \frac{2 \lambda L}{a} $$
		\subsection*{Circular-Aperture Diffraction}
		$$ w = 2 y_1 = 2 L \tan \theta_1 \approx \frac{2.44 \lambda L}{D} $$
		$$ \theta_1 = \frac{1.22 \lambda}{D} \quad \quad y_1 = \frac{1.22 \lambda L}{D} $$
		\subsection*{Interferometers}
		$$ \Delta m = \frac{\Delta L_2}{\lambda/2} $$
		\section*{\large Ray Optics}
		$$ n = \frac{c}{v} $$
		\subsection*{Snell's Law of Refraction}
		$$ n_1 \sin \theta_1 = n_2 \sin \theta_2 $$
		\subsection*{Reflection}
		$$ \theta_r = \theta_i $$
		$$ \theta_c = \sin^{-1}(\frac{n_2}{n_1}) $$
		\subsection*{Ray Tracing}
		$$ m = -\frac{s'}{s} $$
		\subsection*{Thin Lenses}
		$$ \frac{1}{f} = \frac{1}{s} + \frac{1}{s'}$$
		$$ \frac{1}{f} = (n-1)(\frac{1}{R_1} - \frac{1}{R_2})$$
		\subsection*{Spherical Surface}
		$$ \frac{n_1}{s} + \frac{n_2}{s'} = \frac{n_2 - n_1}{R} $$
		\subsection*{Plane Surface}
		$$ R \rightarrow \infty, \quad |s'/s| = n_2/n_1 $$
	\end{multicols*}
\end{document}
